\documentclass{beamer}

\usepackage{hyperref}
\usepackage{amssymb}
\usepackage{latexsym}
\usepackage[utf8]{inputenc}
\DeclareUnicodeCharacter{2218}{\ensuremath{\circ}}
\DeclareUnicodeCharacter{2192}{\ensuremath{\to}}
\DeclareUnicodeCharacter{2200}{\ensuremath{\forall}}
\DeclareUnicodeCharacter{1D52}{\ensuremath{{}^{\text{o}}}}
\DeclareUnicodeCharacter{1D56}{\ensuremath{{}^{\text{p}}}}
\DeclareUnicodeCharacter{03C6}{\ensuremath{\varphi}}
\DeclareUnicodeCharacter{03C8}{\ensuremath{\psi}}
\DeclareUnicodeCharacter{03B1}{\ensuremath{\alpha}}
\DeclareUnicodeCharacter{03B2}{\ensuremath{\beta}}
\DeclareUnicodeCharacter{03B3}{\ensuremath{\gamma}}
\DeclareUnicodeCharacter{03BB}{\ensuremath{\lambda}}
\DeclareUnicodeCharacter{0394}{\ensuremath{\Delta}}
\DeclareUnicodeCharacter{0280}{\ensuremath{{}_0}}
\DeclareUnicodeCharacter{0281}{\ensuremath{{}_1}}
\DeclareUnicodeCharacter{0282}{\ensuremath{{}_2}}
\DeclareUnicodeCharacter{1D30}{\ensuremath{{}^D}}
\DeclareUnicodeCharacter{21D2}{\ensuremath{\Rightarrow}}

\usetheme{Copenhagen}
\usecolortheme{albatross}

\title{Jason Gross' Wishlist for Coq}
\date{POPL 2014 --- Coq Users Meeting}

\setlength{\parskip}{\baselineskip}

\begin{document}
\begin{frame}
\maketitle
\end{frame}

\begin{frame}
\tableofcontents
\end{frame}

\section{Higher Inductive Types}

\begin{frame}[fragile]{Higher Inductive Types}
  \Large
  Higher inductive types are: \pause
  \begin{itemize}
    \item Inductive types \pause
    \item freely generated with higher path structure (non-trivial equalities) \pause
  \end{itemize}
  
  Example: The interval ($0 \leadsto 1$) \pause
\begin{verbatim}
Inductive Interval :=
| zero : Interval
| one  : Interval
| seg  : zero = one. 
\end{verbatim}
\end{frame}

\begin{frame}[fragile]{Higher Inductive Types}{Why?}
  \Large
  Higher inductive types are useful for: \pause
  \begin{itemize}
    \item Homotopy type theory (making basic spaces)\pause
    \item Quotient types \pause
    \item Formalizing version control systems (according to Dan Licata\only<4->{\footnote{``Git as a HIT'', \url{http://dlicata.web.wesleyan.edu/pubs/l13git/git.pdf}}})
    \item Proving functional extensionality
  \end{itemize}
\end{frame}

\begin{frame}[fragile]{Higher Inductive Types}{Proving functional extensionality}
  \Large
\begin{verbatim}
Definition functional_extensionality A B f g
    (H : ∀ x, f x = g x ) → f = g
   := f_equal
       (λ i x ⇒
         match i return B with
           | zero ⇒ f x
           | one  ⇒ g x
           | seg  ⇒ H x
         end)
       seg.
\end{verbatim}
\end{frame}

\begin{frame}[fragile]{Higher Inductive Types}{Proving functional extensionality}
  \Large
\begin{verbatim}
   := match seg in (_ = y)
        return ((λ x ⇒ f x)
                = (λ x ⇒ match y with
                           | zero ⇒ f x
                           | one  ⇒ g x
                           | seg  ⇒ H x
                         end))
      with
        | eq_refl => eq_refl
      end.
\end{verbatim}
\end{frame}

\begin{frame}[fragile]{Higher Inductive Types}{How?}
  \Large
  Note that higher inductive types don't magically give you computational functional extensionality. \pause
  
  You must solve computational functional extensionality to implement computational higher inductive types.
\end{frame}

\begin{frame}[fragile]{Higher Inductive Types}{Current Work}
  \Large
  \begin{itemize}
    \item Yves Bertot's private inductive types;\footnote{\url{http://coq.inria.fr/files/coq5\_submission_3.pdf}} adapted by Matthieu Sozeau
      \begin{itemize}
        \item Easy to implement
      \end{itemize}
  \end{itemize}Note that higher inductive types don't magically give you computational functional extensionality. \pause
  
  You must solve computational functional extensionality to implement computational higher inductive types.
\end{frame}

\begin{frame}
The following features, which I want, are slated for Coq 8.5:
universe polymorphism
parallel processing
tactics in terms
faster sigma types and projections
eta for records

The following are features that I want:
a better story for namespacing (https://coq.inria.fr/bugs/show\_bug.cgi?id=3171)
irrelevant fields/a type of strict hProps/Coq*
built-in CoqMT
judgmental eta for the unit type
a search that searches the entire standard library, and not just currently [Require]d files
a search which is up to unification, rather than up to pattern matching
coercions which don't care about the uniform inheritance condition (see also https://coq.inria.fr/bugs/show\_bug.cgi?id=3115)
built-in MTac or other monadic tactic language
better handling of open terms in ltac, and support for recursing under binders in tactics (maybe fixed with new tactic engine? see also https://coq.inria.fr/bugs/show\_bug.cgi?id=3106 and https://coq.inria.fr/bugs/show\_bug.cgi?id=3102)
easier use of ML plugins (I don't want to have to recompile them myself)
induction-recursion
more uniform support for canonical structures (like I've heard ssr has)
support for computational parametricity
better coinduction (should be compositional, maybe based on copatterns or something)
size/type-based termination
faster rewrite
support for reflective simplification (maybe a native reifier which runs at type inference time, and a special type in the stdlib or something for syntax)
automatic generation of the equivalence between record types and nested sigma types
ability to write theorems that apply to all records, which are specialized at type-inference time (a la typeclasses or mtac)
rewrite that alternates simpl and argument inference
rewrite which matches the head by pattern matching and the rest by unification
variant of @? patterns for [pattern]ing on things other than bound indices and parameters, heuristically (https://coq.inria.fr/bugs/show\_bug.cgi?id=3148)
a variant of [Hint Rewrite] which infers arguments based on pattern matching then runs [simpl] on the hypothesis, then rewrites with the simplified hypothesis
'where' clauses in records should permit abbreviations (https://coq.inria.fr/bugs/show\_bug.cgi?id=3066)
variant of [abstract] which finishes the subproof with [Defined] rather than [Qed] (and another variant which finishes it with [Defined] and then runs [Global Opaque] on the constant)
allow overriding symmetry, reflexivity (https://coq.inria.fr/bugs/show\_bug.cgi?id=3113)
etransitivity should take an optional term with holes (https://coq.inria.fr/bugs/show\_bug.cgi?id=3065)
where clauses in records should support (only parsing) (https://coq.inria.fr/bugs/show\_bug.cgi?id=3067)
support for simultaneous generation of terms binding scopes (https://coq.inria.fr/bugs/show\_bug.cgi?id=3123)
have a [function\_scope] like [type\_scope] (https://coq.inria.fr/bugs/show\_bug.cgi?id=3080)
judgmental eta for all inductive types (https://coq.inria.fr/bugs/show\_bug.cgi?id=3119)
more judgmental commutation rules for [match] (see https://coq.inria.fr/bugs/show\_bug.cgi?id=3191 and related bugs)
support for computational higher inductive types
support for computational univalence
computational functional extensionality
better handling (speed-wise) of large terms and types (native projections might fix this)
support for explicit universe level variables (without loosing the default of typical ambiguity)
eq\_sym (eq\_sym p) should be judgmentally equal to p


\end{frame}
\end{document}