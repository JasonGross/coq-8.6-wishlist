\documentclass{beamer}

\usepackage{hyperref}
\usepackage{amssymb}
\usepackage{latexsym}
\usepackage[utf8]{inputenc}
\DeclareUnicodeCharacter{2218}{\ensuremath{\circ}}
\DeclareUnicodeCharacter{2192}{\ensuremath{\to}}
\DeclareUnicodeCharacter{2200}{\ensuremath{\forall}}
\DeclareUnicodeCharacter{1D52}{\ensuremath{{}^{\text{o}}}}
\DeclareUnicodeCharacter{1D56}{\ensuremath{{}^{\text{p}}}}
\DeclareUnicodeCharacter{03C6}{\ensuremath{\varphi}}
\DeclareUnicodeCharacter{03C8}{\ensuremath{\psi}}
\DeclareUnicodeCharacter{03B1}{\ensuremath{\alpha}}
\DeclareUnicodeCharacter{03B2}{\ensuremath{\beta}}
\DeclareUnicodeCharacter{03B3}{\ensuremath{\gamma}}
\DeclareUnicodeCharacter{03BB}{\ensuremath{\lambda}}
\DeclareUnicodeCharacter{0394}{\ensuremath{\Delta}}
\DeclareUnicodeCharacter{0280}{\ensuremath{{}_0}}
\DeclareUnicodeCharacter{0281}{\ensuremath{{}_1}}
\DeclareUnicodeCharacter{0282}{\ensuremath{{}_2}}
\DeclareUnicodeCharacter{1D30}{\ensuremath{{}^D}}
\DeclareUnicodeCharacter{21D2}{\ensuremath{\Rightarrow}}
\DeclareUnicodeCharacter{2115}{\ensuremath{\mathbb{N}}}
\DeclareUnicodeCharacter{2261}{\ensuremath{\equiv}}

% http://www-ljk.imag.fr/membres/Jerome.Lelong/latex/appendixnumberbeamer.sty
% Reference: http://tex.stackexchange.com/questions/2541/beamer-frame-numbering-in-appendix
\usepackage{appendixnumberbeamer}

\usetheme{Copenhagen}
\usecolortheme{albatross}

\title{Jason Gross' Wishlist for Coq}
\date{POPL 2014 --- Coq Users Meeting}

\setlength{\parskip}{0.5\baselineskip}

\begin{document}
\begin{frame}
\maketitle
\end{frame}

\begin{frame}
\tableofcontents
\end{frame}


\section{More Powerful Judgmental Equality}

\begin{frame}{Judgmental Equality}
  \Large
  \begin{center}
    More Powerful Judgmental Equality
  \end{center} \pause
  
  Warning: Some of my proposals get rather insane, so the further on in this section they are, the more grains of salt you should be taking them with.
\end{frame}

\begin{frame}{Judgmental Equality}{My Wishes: \texorpdfstring{$\eta$}{η} for records}
  \Large
  \begin{center}
    $\eta$ for records
  \end{center}
  
  Implemented by Matthieu Sozeau; in 8.5, I can now have $(\mathcal C^{\text{op}})^\text{op} \equiv \mathcal C$ for categories $\mathcal C$! \pause
  
  It would still be nice to have
  \begin{center}
  \texttt{$\forall$ x y :\ unit, x $\equiv$ tt $\equiv$ y}.
  \end{center}
\end{frame}

\begin{frame}[fragile]{Judgmental Equality}{My Wishes: \texorpdfstring{$\eta$}{η} for inductives}
  \Large
  \begin{center}
    $\eta$ for inductive types
  \end{center}
  I want
\begin{verbatim}
∀ A B (x : A + B),
  match x with
    | inl x' ⇒ inl x'
    | inr x' ⇒ inr x'
  end ≡ x
\end{verbatim}
\end{frame}

\begin{frame}[fragile]{Judgmental Equality}{My Wishes: \texorpdfstring{$\eta$}{η} for inductives}
  \Large
  \begin{center}
    $\eta$ for inductive types
  \end{center}
  I want
\begin{verbatim}
∀ A (x y : A) (p : x = y),
  match p in (_ = y') return (x = y') with
    | eq_refl ⇒ eq_refl
  end ≡ p

\end{verbatim}
\end{frame}

\begin{frame}[fragile]{Judgmental Equality}{My Wishes: Computation Rules for \texorpdfstring{\texttt{match}}{match}}
  \Large
  \begin{center}
    More computation rules for \texttt{match}
  \end{center}
  I want a \texttt{match} to eat up unused arguments:
\begin{verbatim}
  match p as p' in (T x _)
    return (T' x p' → T'' x p')
  with
    | con1 ⇒ (λ _ ⇒ val1)
    ...
  end y
  ≡
\end{verbatim}
\end{frame}

\begin{frame}[fragile]{Judgmental Equality}{My Wishes: Computation Rules for \texorpdfstring{\texttt{match}}{match}}
  \Large
  \begin{center}
    More computation rules for \texttt{match}
  \end{center}
  I want a \texttt{match} to eat up unused arguments:
\begin{verbatim}
  ≡
  match p as p' in (T x _)
    return (T'' x p')
  with
    | con1 ⇒ val1
    ...
  end
\end{verbatim}
\end{frame}

\begin{frame}[fragile]{Judgmental Equality}{My Wishes: Computation Rules for \texorpdfstring{\texttt{match}}{match}}
  \Large
  \begin{center}
    More computation rules for \texttt{match}
  \end{center}
  I want \texttt{match}es to distribute over arrows
\begin{verbatim}
  match p as p' in (T x _)
    return (∀ y : T', T'' x p' y)
  with
    | con1 ⇒ f1
    ...
  end
  ≡
\end{verbatim}
\end{frame}

\begin{frame}[fragile]{Judgmental Equality}{My Wishes: Computation Rules for \texorpdfstring{\texttt{match}}{match}}
  \Large
  \begin{center}
    More computation rules for \texttt{match}
  \end{center}
  I want \texttt{match}es to distribute over arrows
\begin{verbatim}
  ≡
  (λ y : T' ⇒ 
      match p as p' in (T x _)
        return (T'' x p' y)
      with
        | con1 ⇒ f1 y
        ...
      end)
\end{verbatim}
\end{frame}

\begin{frame}[fragile]{Judgmental Equality}{My Wishes: Computation Rules for \texorpdfstring{\texttt{match}}{match}}
  \Large
  \begin{center}
    More computation rules for \texttt{match}
  \end{center}
  I want a \texttt{match} whose branches unify to disappear (if the return type is constant)
\begin{verbatim}
  match p return T with
    | _ ⇒ val
  end ≡ val
\end{verbatim}
\end{frame}

\begin{frame}[fragile]{Judgmental Equality}{My Wishes: Computation Rules for \texorpdfstring{\texttt{match}}{match}}
  \Large
  \begin{center}
    More computation rules for \texttt{match}
  \end{center}
  I want \texttt{match}es to distribute over inductive types (when the branches unify appropriately)
\begin{verbatim}
  match p as p' in (T x _)
    return (T' (f x p'))
  with
    | con1 ⇒ Build_T' _ con1 val1
    ...
  end
  ≡
\end{verbatim}
\end{frame}

\begin{frame}[fragile]{Judgmental Equality}{My Wishes: Computation Rules for \texorpdfstring{\texttt{match}}{match}}
  \Large
  \begin{center}
    More computation rules for \texttt{match}
  \end{center}
  I want \texttt{match}es to distribute over inductive types (when the branches unify appropriately)
\begin{verbatim}
  ≡
  Build_T'
    (match p with | con1 ⇒ f _ con1 | ... end)
    (match p with | con1 ⇒ con1 | ... end)
    (match p with | con1 ⇒ val1 | ... end)
    
    
\end{verbatim}
\end{frame}

\begin{frame}[fragile]{Judgmental Equality}{My Wishes: Computation Rules for \texorpdfstring{\texttt{match}}{match}}
  \Large
  \begin{center}
    More computation rules for \texttt{match}
  \end{center}
  I want \texttt{match}es on \texttt{match}es to reduce to \texttt{match}es which return \texttt{match}es
\begin{verbatim}
  match (match ... with ... end) with ... ⇒ ... end
  ≡
  match ... with ... ⇒ ... (match ... with ... end) end
\end{verbatim}
\end{frame}

\begin{frame}[fragile]{Judgmental Equality}{My Wishes: Judgmental Groupoid Laws}
  \Large
  \begin{center}
    Judgmental Groupoid Laws
  \end{center}
  I want (the option of) \texttt{Type}s to be strict $\infty$-groupoids
  
  \begin{align*}
    (p^{-1})^{-1} & \equiv p & \text{($p^{-1}$ is \texttt{eq\_sym} $p$)} \\
    p \circ (q \circ r) & \equiv (p \circ q) \circ r\quad & \text{($p \circ q$ is \texttt{eq\_trans} $p$ $q$)} \\
    p \circ 1 & \equiv p \equiv 1 \circ p & \text{(1 is \texttt{eq\_refl})}
  \end{align*}
\end{frame}

\begin{frame}[fragile]{Judgmental Equality}{My Wishes: Axiom K-based Pattern Matching When It's Provable}
  \Large
  \begin{center}
    K-Based Pattern Matching
  \end{center}
  I want K-based pattern matching on types which Coq can infer are hSets (satisfy uniqueness of identity proofs, and therefore K), any maybe for types where I can prove K. \pause  Alternatively, maybe a ``strict 0-truncation'' operator, and support for K there. \pause
  
  Proposal by Pierre Corbineau: ``The K axiom in Coq (almost) for free''\footnote<2->{\url{http://coq.inria.fr/files/adt-2fev10-corbineau.pdf}}
\end{frame}

\begin{frame}[fragile]{Judgmental Equality}{My Wishes: Irrelevant Types}
  \Large
  \begin{center}
    Irrelevant Types
  \end{center}
  I want types with judgmental (proof) irrelevance, like dotted fields in Agda. \pause  These are strict hProps.  \pause
  
  Current work: Miquel's implicit calculus of constructions (ICC), B. Barras and B. Bernardo's decidable version (ICC*)
\end{frame}

\begin{frame}[fragile]{Judgmental Equality}{My Wishes: Reflection When We Can Have It}
  \Large
  \begin{center}
    Limited Equality Reflection
  \end{center}
  I want equality reflection whenever it doesn't break things
\begin{verbatim}
(∀ (x : T) (pf : x = x), pf = eq_refl)
→ ∀ (x : T) (pf : x = x), pf ≡ eq_refl    
\end{verbatim}
\pause
  (What's a general rule?  Inductive type families with one constructor which are all provably equal to that constructor?)
\end{frame}

\begin{frame}[fragile]{Judgmental Equality}{My Wishes: Postulating Judgmental Equality}
  \Large
  \begin{center}
    Postulating Judgmental Equality?
  \end{center}
  Voevodsky suggests (and Dan Grayson has worked on implementing) having two equality types, a non-fibrant reflected equality type, and a fibrant intensional equality type.  Perhaps Coq should go this route one day?
\end{frame}

\begin{frame}{Judgmental Equality}{My Wishes}
  \Large
  I also want:
  \begin{itemize}
    \item \texttt{($\lambda$ x y $\Longrightarrow$ x + y) $\equiv$ ($\lambda$ x y $\Longrightarrow$ y + x)} (done in CoqMT by Pierre-Yves Strub)
    \item ability to add computation rules for axioms \pause
      \begin{itemize} \large
        \item univalence \pause
        \item functional extensionality \pause
        \item higher inductive types \pause
        \item internalized parametricity
      \end{itemize}
  \end{itemize}
\end{frame}

\begin{frame}{Judgmental Equality}{Implementation Properties}
  \Large
  \begin{itemize}
    \item should be optional extensions \pause
    \item should be customizable, with plug-ins or flags or both \pause
    \item type-checking should still be decidable
  \end{itemize}
\end{frame}

\begin{frame}{Judgmental Equality}{My Wishes: Why?}
  \Large
  Why? \pause
  
  Theorem proving is easier when the type-checker does more work for me. \pause
  
  And it seems like an interesting system to play with.
\end{frame}

\section{Higher Inductive Types}
\subsection{What are they?}

\begin{frame}[fragile]{Higher Inductive Types}
  \Large
  Higher inductive types are: \pause
  \begin{itemize}
    \item Inductive types \pause
    \item freely generated with higher path structure (non-trivial equalities) \pause
  \end{itemize}
  Example: The interval ($0 \leadsto 1$) \pause
\begin{verbatim}
Inductive Interval :=
| zero : Interval
| one  : Interval
| seg  : zero = one. 
\end{verbatim}
\end{frame}

\subsection{How are they useful?}

\begin{frame}[fragile]{Higher Inductive Types}{Why?}
  \Large
  Higher inductive types are useful for: \pause
  \begin{itemize}
    \item Homotopy type theory (making basic spaces)\pause
    \item Quotient types \pause
    \item Formalizing version control systems (according to Dan Licata%\only<4->{%
    \footnote<4->{``Git as a HIT'', \url{http://dlicata.web.wesleyan.edu/pubs/l13git/git.pdf}}%
    %}%
    ) \pause
    \item Proving functional extensionality
  \end{itemize}
\end{frame}

\begin{frame}[fragile]{Higher Inductive Types}{Proving functional extensionality}
  \Large
\begin{verbatim}
Definition functional_extensionality A B f g
     : (∀ x, f x = g x) → f = g
   := λ H ⇒ f_equal
             (λ i x ⇒
               match i return B with
                 | zero ⇒ f x
                 | one  ⇒ g x
                 | seg  ⇒ H x
               end)
             seg.
\end{verbatim}
\end{frame}

\begin{frame}[fragile]{Higher Inductive Types}{Proving functional extensionality}
  \Large
\begin{verbatim}
   := match seg in (_ = y)
        return ((λ x ⇒ f x)
                = (λ x ⇒ match y with
                           | zero ⇒ f x
                           | one  ⇒ g x
                           | seg  ⇒ H x
                          end))
      with
        | eq_refl => eq_refl
      end.
\end{verbatim}
\end{frame}

\subsection{Implementation}

\begin{frame}{Higher Inductive Types}{How?}
  \Large
  Note that higher inductive types don't magically give you computational functional extensionality. \pause
  
  You must solve computational functional extensionality to implement computational HITs. \pause
  
  (Similar story for implementing computational univalence, another feature on my wishlist.) \pause
  
  Breaks canonicity \pause (jugdmentally), \pause preserves it up to propositional equality? {\normalsize (conjecture by Voevodsky for UA)}
\end{frame}

\begin{frame}{Higher Inductive Types}{Current Work}
  \Large
  \begin{itemize}
    \item Yves Bertot's private inductive types;\footnote{\url{http://coq.inria.fr/files/coq5\_submission_3.pdf}} adapted by Matthieu Sozeau \pause
      \begin{itemize} \large
        \item Comparatively easy to implement \pause
        \item Allows one to disable pattern matching on inductive types outside a module, which is sufficient to implement a trick by Dan Licata\footnote<3->{\url{http://homotopytypetheory.org/2011/04/23/running-circles-around-in-your-proof-assistant/}} \pause
        \item Equalities are axioms; not computational \pause
        \item Only eliminators, no pattern matching \pause
      \end{itemize}
    \item Burno Barras has some partial work that's more computational\footnote<6->{\url{https://github.com/barras/coq/tree/hit}}
  \end{itemize}
\end{frame}

\begin{frame}{Higher Inductive Types}{My Wishes}
  \Large
  I want: \pause
  \begin{itemize}
	\item to be able to define and pattern match on higher inductive types \pause
	\item all tactics should support HITs \pause
	\item judgmental reduction rules for matching on paths from HITs \pause
	\item equality should not be special \pause
	  \begin{itemize} \large
	    \item typechecker should not depend on standard library \pause
	    \item c.f. proposal for pattern matching justifying K\footnote<7->{``The K axiom in Coq (almost) for free'' \url{http://coq.inria.fr/files/adt-2fev10-corbineau.pdf}}
	  \end{itemize}
  \end{itemize}
\end{frame}

\begin{frame}[fragile]{Higher Inductive Types (without equality in the kernel)}{Possible Generalization (I)}
  \Large
  \begin{itemize}
    \item If equality isn't special, then HITs can put inhabitants in arbitrary types \pause
    \item BAD, if it allows us to give a proof of \texttt{False}
  \end{itemize}
\begin{verbatim}



\end{verbatim}
\end{frame}

\begin{frame}[fragile]{Higher Inductive Types (without equality in the kernel)}{Possible Generalization (I)}
  \Large
  \begin{itemize}
    \item If equality isn't special, then HITs can put inhabitants in arbitrary types
    \item BAD, if it allows us to give a proof of \texttt{False}
  \end{itemize}
\begin{verbatim}
Inductive BAD : Set :=
| silly : BAD
| terrible : False.
\end{verbatim}
\end{frame}

\begin{frame}{Higher Inductive Types (without equality in the kernel)}{Possible Generalization (I)}
  \Large
  \begin{itemize}
    \item If equality isn't special, then HITs can put inhabitants in arbitrary types
    \item BAD, if it allows us to give a proof of \texttt{False}
    \item Idea: Require providing an inhabitant of the appropriate type family \pause
      \begin{itemize} \large
        \item Used to pick out which branch of pattern matching to use \pause
        \item Simply reduces when the provided term sits in the right type (not just right type family)
      \end{itemize}
  \end{itemize}
\end{frame}

\begin{frame}[fragile]{Higher Inductive Types (without equality in the kernel)}{Possible Generalization (I)}
  \Large
\begin{verbatim}
Inductive Interval : Type :=
| zero : Interval
| one : Interval
| seg : zero = one
and picking
| seg : zero = _ := eq_refl.
\end{verbatim}
\end{frame}

\begin{frame}[fragile]{Higher Inductive Types (without equality in the kernel)}{Possible Generalization (II)}
  \Large
\begin{verbatim}
Inductive _==_ `(x : A) : ∀ {B}, B → Type :=
| refl1 : x == x
| refl2 : x == x.
Inductive foo : Type :=
| bar : nat → foo
| proof1 : ∀ (n : ℕ), bar 2 == bar (S (S n))
| proof2 : ∀ (n : ℕ), bar 0 == bar 1
and picking
| proof1 : ∀ n, bar 2 == _ := λ n ⇒ refl1
| proof2 : ∀ n, bar 0 == _ := λ n ⇒ refl2.
\end{verbatim}
\end{frame}

\begin{frame}{Higher Inductive Types (without equality in the kernel)}{Possible Generalization (III)}
  \Large
  Mike Shulman tells me this might be saying that a generalized higher inductive type is a polynomial functor $F$ together with an object of $F(1)$. \pause
  
  We still need computation rules for this.  \pause (See \autoref{sec:comp-hit}) \pause
  
  Also an implementation, and justification of consistency.
\end{frame}

\section{The Rest of my Wishlist}


\begin{frame}{The Rest of my Wishlist (I)}
This was just a small (but important) part of my wishlist.  The rest:

\begin{itemize}
  \item
    a better story for namespacing\footnote{\url{https://coq.inria.fr/bugs/show\_bug.cgi?id=3171}}
  \item 
    induction-recursion, induction-induction, etc.
  \item
    very dependent types, insanely dependent types ($\Sigma$ as $\Pi$)\footnote{\url{https://github.com/UlfNorell/insane}, ``Formal Objects in Type Theory Using Very Dependent Types'' \url{http://citeseerx.ist.psu.edu/viewdoc/download?doi=10.1.1.39.4169&rep=rep1&type=pdf}}
  \item 
    better coinduction (should be compositional, maybe based on copatterns)
  \item 
    size/type-based termination
  \item 
    support for explicit universe level variables (without loosing the default of typical ambiguity)
\end{itemize}
\end{frame}

\begin{frame}{The Rest of my Wishlist (II)}
\begin{itemize}
  \item
    parallel version of \texttt{all: solve} when there are no evars in the goal
  \item 
    a search that searches the entire standard library, and not just currently \texttt{Require}d files
  \item 
    a search which is up to unification, rather than up to pattern matching
  \item 
    coercions that don't care about the uniform inheritance condition\footnote{\url{https://coq.inria.fr/bugs/show\_bug.cgi?id=3115}}
  \item 
    faster rewrite
  \item 
    automatic generation of the equivalence between record types and nested sigma types
  \item 
    ability to write theorems that apply to all records, which are specialized at type-inference time (a la typeclasses or mtac)
\end{itemize}
\end{frame}

\begin{frame}{The Rest of my Wishlist (II)}
\begin{itemize}
  \item
    notations should be able to pick a meaning based on the type of their constituents (but must have a consistent scope for each term across all meanings) (can currently be hacked with boilerplate, typeclasses, and \texttt{\$(...)\$} to remove the typeclasses)\footnote{\url{https://coq.inria.fr/bugs/show\_bug.cgi?id=3090}}
  \item 
    better handling of open terms in Ltac, and support for recursing under binders in tactics (maybe fixed with new tactic engine?)\footnote{\url{https://coq.inria.fr/bugs/show\_bug.cgi?id=3106} and \url{https://coq.inria.fr/bugs/show\_bug.cgi?id=3102}}
  \item 
    easier use of ML plugins (I don't want to have to recompile them myself)
\end{itemize}
\end{frame}

\begin{frame}{The Rest of my Wishlist (III)}
\begin{itemize}
  \item 
    more uniform support for canonical structures (like ssr has)
  \item 
    support for reflective simplification (maybe a native reifier which runs at type inference time, and a special type in the stdlib or something for syntax)
  \item 
    rewrite that alternates simpl and argument inference
  \item 
    rewrite which matches the head by pattern matching and the rest by unification
  \item 
    variant of @? patterns for [pattern]ing on things other than bound indices and parameters, heuristically\footnote{\url{https://coq.inria.fr/bugs/show\_bug.cgi?id=3148}}
  \item 
    have a \texttt{function\_scope} like \texttt{type\_scope}\footnote{\url{https://coq.inria.fr/bugs/show\_bug.cgi?id=3080}}
\end{itemize}
\end{frame}

\begin{frame}{The Rest of my Wishlist (IV)}
\begin{itemize}
  \item 
    a variant of \texttt{Hint Rewrite} which infers arguments based on pattern matching then runs \texttt{simpl} on the hypothesis, then rewrites with the simplified hypothesis
  \item 
    'where' clauses in records should permit abbreviations\footnote{\url{https://coq.inria.fr/bugs/show\_bug.cgi?id=3066}}
  \item 
    variant of \texttt{abstract} which finishes the subproof with \texttt{Defined} rather than \texttt{Qed} (and another variant which finishes it with \texttt{Defined} and then runs \texttt{Global Opaque} on the constant)
  \item 
    allow overriding symmetry, reflexivity\footnote{\url{https://coq.inria.fr/bugs/show\_bug.cgi?id=3113}}
\end{itemize}
\end{frame}

\begin{frame}{The Rest of my Wishlist (V)}
\begin{itemize}
  \item 
    etransitivity should take an optional term with holes\footnote{\url{https://coq.inria.fr/bugs/show\_bug.cgi?id=3065}}
  \item 
    where clauses in records should support \texttt{(only parsing)}\footnote{\url{https://coq.inria.fr/bugs/show\_bug.cgi?id=3067}}
  \item 
    support for simultaneous generation of terms binding scopes\footnote{\url{https://coq.inria.fr/bugs/show\_bug.cgi?id=3123}}
  \item 
    better handling (speed-wise) of large terms and types (native projections might fix this)
\end{itemize}
\end{frame}

\begin{frame}
\Huge\begin{center}
Thanks!

Questions?
\end{center}
\end{frame}

\appendix


\section{Computation Rules for HITs} \label{sec:comp-hit}

\begin{frame}[fragile]{Computation Rules for HITs}{Proposed computation rule for HITs}
\Large
Given a higher inductive type \verb|T| and a path constructor \verb|p : a = b|, we should have
\begin{verbatim}
match p in (_ = y)
  return (P (fixmatch {h} y with 
               | a => c
               | b => d
               | p => f
             end)) with
  | eq_refl => g
end
≡
\end{verbatim}
\end{frame}


\begin{frame}[fragile]{Computation Rules for HITs}{Proposed computation rule for HITs}
\Large
Given a higher inductive type \verb|T| and a path constructor \verb|p : a = b|, we should have
\begin{verbatim}
≡
match f in (_ = y) return (P y) with
  | eq_refl => g
end





\end{verbatim}
\end{frame}

\end{document}